%-----------------------------------------------------------------------------%
\chapter{PENUTUP}
\label{bab:3}
%-----------------------------------------------------------------------------%
Pada bab ini dipaparkan kesimpulan dan saran dari pelaksanaan kerja praktik yang dilakukan di Pondok Modern Al-Ghozali. Pembahasan diawali dengan kesimpulan secara umum, kemudian dilanjutkan dengan penyampaian saran yang berkaitan dengan pelaksanaan kerja praktik.
%-----------------------------------------------------------------------------%
\section{Kesimpulan}
\label{sec:kesimpulan}
%-----------------------------------------------------------------------------%
Pelaksanaan kerja praktik di Pondok Modern Al-Ghozali memberikan pengalaman yang berharga bagi pelaksana, baik dari sisi teknis maupun non-teknis. Selama periode kerja praktik, pelaksana terlibat langsung dalam proses pengembangan sistem dashboard absensi guru dan sistem perizinan berbasis web yang digunakan secara nyata dalam kegiatan operasional sekolah.

Dari sisi teknis, pelaksana memperoleh pemahaman yang lebih mendalam mengenai pengembangan \textit{backend} aplikasi web menggunakan \textit{framework} Django, pengolahan data \textit{fingerprint} dalam format XLS, serta implementasi fitur dashboard statistik dan integrasi notifikasi WhatsApp. Sistem yang dibangun berhasil diimplementasikan secara \textit{live} dan terbukti membantu pihak sekolah dalam mengelola data kehadiran dan perizinan guru secara lebih efisien.

Dari sisi non-teknis, peran pelaksana sebagai Project Manager memberikan wawasan mengenai siklus hidup pengembangan perangkat lunak, mulai dari perencanaan, pelaksanaan, hingga evaluasi. Pelaksana juga belajar mengenai pentingnya komunikasi dengan \textit{stakeholder}, pengelolaan waktu, serta pengambilan keputusan dalam lingkungan kerja profesional. Secara keseluruhan, kerja praktik ini memberikan pengalaman nyata yang sangat relevan sebagai bekal pelaksana untuk memasuki dunia kerja di bidang teknologi informasi.

%-----------------------------------------------------------------------------%
\section{Saran}
\label{sec:saran}
%-----------------------------------------------------------------------------%
Berdasarkan pelaksanaan kerja praktik yang telah dilakukan, terdapat beberapa saran yang dapat disampaikan kepada pelaksana kerja praktik berikutnya, pihak Pondok Modern Al-Ghozali sebagai tempat kerja praktik, serta Fakultas Ilmu Komputer Universitas Indonesia.

\subsection{Pelaksana Kerja Praktik Berikutnya}
Kerja praktik merupakan kesempatan yang baik bagi mahasiswa untuk menerapkan ilmu yang diperoleh selama perkuliahan ke dalam permasalahan nyata di lapangan. Oleh karena itu, pelaksana kerja praktik berikutnya disarankan untuk mempersiapkan diri dengan baik, terutama dalam hal manajemen waktu antara kewajiban akademik dan tanggung jawab kerja praktik.

Selain itu, pelaksana berikutnya disarankan untuk aktif berkomunikasi dengan mentor dan \textit{stakeholder} ketika menghadapi kendala, baik teknis maupun non-teknis. Komunikasi yang baik dapat membantu mempercepat penyelesaian masalah serta mencegah terjadinya kesalahpahaman yang dapat menghambat jalannya proyek.

\subsection{Tempat Kerja Praktik}
Bagi pihak Pondok Modern Al-Ghozali, sistem yang telah dikembangkan masih memiliki peluang untuk dikembangkan lebih lanjut. Salah satu pengembangan yang dapat dipertimbangkan adalah integrasi langsung antara sistem dengan mesin \textit{fingerprint} melalui \textit{Application Programming Interface} (API), sehingga proses pengambilan data dapat dilakukan secara otomatis tanpa perlu unggah berkas secara manual.

Selain itu, pihak sekolah disarankan untuk melakukan pencadangan data secara rutin guna menjaga keamanan dan keberlanjutan sistem dalam jangka panjang. Pengembangan fitur tambahan sesuai dengan kebutuhan operasional di masa mendatang juga dapat menjadi \textit{future work} dari sistem yang telah dibangun.

\subsection{Fasilkom UI}
Fakultas Ilmu Komputer Universitas Indonesia diharapkan dapat terus mendukung pelaksanaan kerja praktik dengan menjalin kerja sama yang lebih luas dengan berbagai institusi dan lembaga pendidikan. Kerja sama tersebut dapat membuka lebih banyak peluang bagi mahasiswa untuk mendapatkan tempat kerja praktik yang relevan dengan bidang keilmuan.

Selain itu, dukungan dari fakultas dalam bentuk panduan dan fleksibilitas pelaksanaan kerja praktik sangat membantu mahasiswa untuk fokus memperoleh pengalaman profesional tanpa mengabaikan kewajiban akademik yang sedang dijalani.