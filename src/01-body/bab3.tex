%-----------------------------------------------------------------------------%
\chapter{PENUTUP}
\label{bab:3}
%-----------------------------------------------------------------------------%

%-----------------------------------------------------------------------------%
\section{Kesimpulan}
\label{sec:kesimpulan}
%-----------------------------------------------------------------------------%
Berdasarkan pelaksanaan kerja praktik yang telah dilakukan, sistem dashboard absensi dan perizinan berhasil dibangun dan diimplementasikan secara langsung di lingkungan Pondok Modern Al-Ghozali.
Aplikasi yang dikembangkan terbukti mampu membantu pihak sekolah, khususnya admin, dalam mengelola ribuan data kehadiran guru setiap bulan secara lebih efisien dan terstruktur.

Selain itu, peran pelaksana sebagai \textit{Project Manager} memberikan pemahaman yang mendalam mengenai siklus hidup pengembangan perangkat lunak, mulai dari tahap perencanaan, implementasi, hingga evaluasi sistem.
Digitalisasi proses absensi dan perizinan ini membawa dampak positif terhadap efektivitas administrasi serta mempercepat alur penyampaian informasi di lingkungan sekolah.

%-----------------------------------------------------------------------------%
\section{Saran}
\label{sec:saran}
%-----------------------------------------------------------------------------%
Berdasarkan hasil pelaksanaan kerja praktik, terdapat beberapa saran yang dapat dipertimbangkan untuk pengembangan selanjutnya.
Pelaksana berikutnya disarankan untuk mengembangkan integrasi langsung antara sistem dengan mesin \textit{fingerprint} melalui API agar proses pengambilan data kehadiran dapat dilakukan secara otomatis tanpa unggah file manual.

Selain itu, pihak sekolah disarankan untuk melakukan pencadangan data secara rutin melalui fitur yang telah tersedia guna mengantisipasi kehilangan data.
Fakultas Ilmu Komputer Universitas Indonesia juga diharapkan dapat terus menjalin kerja sama dengan lembaga pendidikan lokal agar implementasi teknologi informasi dapat memberikan manfaat yang lebih luas.
