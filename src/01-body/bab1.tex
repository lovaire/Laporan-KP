%----------------------------------------------------------------------------%
\chapter{PENDAHULUAN}
\label{bab:pendahuluan}
%----------------------------------------------------------------------------%

Bab ini menjelaskan latar belakang pelaksanaan kerja praktik, proses pencarian tempat kerja praktik, serta gambaran umum mengenai tempat dan posisi pelaksanaan kerja praktik.

%----------------------------------------------------------------------------%
\section{Proses Pencarian Kerja Praktik}
\label{sec:prosesKP}
%----------------------------------------------------------------------------%

Proses pencarian Kerja Praktik bermula dari inisiatif pelaksana untuk melaksanakan kegiatan kerja praktik di lingkungan yang telah dikenal sebelumnya. Pondok Modern Al-Ghozali merupakan institusi pendidikan tempat pelaksana menempuh pendidikan pada jenjang sebelumnya, sehingga pelaksana telah memahami secara umum kondisi, kebutuhan, serta tantangan yang dihadapi oleh institusi tersebut, khususnya dalam bidang teknologi informasi.

Seiring dengan perkembangan teknologi dan meningkatnya kebutuhan akan sistem informasi yang terintegrasi, Pondok Modern Al-Ghozali sedang gencar melakukan upaya digitalisasi pada berbagai aspek administrasi dan operasional sekolah. Beberapa proses yang sebelumnya masih dilakukan secara manual mulai diarahkan untuk beralih ke sistem berbasis digital guna meningkatkan efisiensi, akurasi data, serta kemudahan pengelolaan informasi. Kondisi tersebut menjadi peluang sekaligus motivasi bagi pelaksana untuk mengajukan kerja praktik di institusi tersebut.

Berdasarkan latar belakang tersebut, pelaksana secara mandiri mengajukan permohonan kerja praktik kepada pihak Pondok Modern Al-Ghozali dengan membawa gagasan pengembangan sistem informasi yang dapat mendukung proses digitalisasi sekolah, khususnya pada pengelolaan absensi dan perizinan guru. Pengajuan kerja praktik ini disampaikan secara langsung kepada pimpinan sekolah dan kepala bidang terkait dengan menjelaskan tujuan kerja praktik, ruang lingkup pekerjaan, serta manfaat yang diharapkan dapat diperoleh oleh pihak sekolah.

Setelah melalui proses diskusi dan pertimbangan, pihak pimpinan Pondok Modern Al-Ghozali menyambut baik inisiatif yang diajukan oleh pelaksana. Usulan kerja praktik dinilai sejalan dengan kebutuhan sekolah dalam meningkatkan pemanfaatan teknologi informasi. Oleh karena itu, permohonan kerja praktik yang diajukan oleh pelaksana kemudian disetujui secara resmi oleh pihak sekolah.

%----------------------------------------------------------------------------%
\section{Tempat Kerja Praktik}
\label{sec:tempatKP}
%----------------------------------------------------------------------------%

%----------------------------------------------------------------------------%
\subsection{Profil Tempat Kerja Praktik}
\label{sec:profilTempatKP}
%----------------------------------------------------------------------------%

Pondok Modern Al-Ghozali adalah lembaga pendidikan formal yang berlokasi di Gunungsindur, Bogor. Lembaga ini mengelola jenjang pendidikan MI, SMP, dan SMA Islam. Karena tuntutan zaman yang semakin modern, kini Pondok Modern Al-Ghozali sedang gencar melakukan upaya digitalisasi dalam beberapa aspek, mulai dari kegiatan belajar mengajar hingga urusan administrasi.

Sebagai Pondok terkemuka di wilayah kecamatan Gunungsindur, Pondok Modern Al-Ghozali berkomitmen untuk mendorong upaya digitalisasi sebagai bentuk kemajuan peradaban umat. Dengan adanya upaya digitalisasi pada Pondok Modern Al-Ghozali ini, diharapkan kedepannya untuk dapat mendukung visi Pondok Modern Al-Ghozali, yakni "Terwujudnya pondok pesantren yang unggul, berkualitas dan relevan menuju terbentuknya sumber daya manusia yang islami dan excellent yang memiliki kecerdasan spiritual, intelektual, emosional, kreatif dan nilai-nilai profesionalisme yang berlandaskan Al-Quran, Hadist serta Pancasila dan UUD 1945."

%----------------------------------------------------------------------------%
\subsection{Posisi Penempatan Pelaksana Kerja Praktik}
\label{sec:posisiKP}
%----------------------------------------------------------------------------%

Dalam mendukung upaya pengembangan digitalisasinya, Pondok Modern Al-Ghozali membentuk tim khusus untuk bidang ini, yang juga tergabung pada Divisi Media dan Syiar pondok. Pada kegiatan Kerja Praktik ini, pelaksana bersama kelompok masuk pada Tim Pengembang TI Pondok Modern Al-Ghozali.

Pelaksana menempati posisi sebagai Project Manager sekaligus Backend Developer. Dalam peran sebagai Project Manager, pelaksana bertanggung jawab mengkoordinasikan alur pengembangan sistem, menyusun perencanaan sprint, serta memastikan setiap kebutuhan stakeholder dapat diterjemahkan ke dalam fitur sistem. Sementara itu, sebagai Backend Developer, pelaksana mengembangkan logika bisnis aplikasi, mengelola struktur database, serta memastikan integrasi antar modul sistem berjalan dengan baik. Kombinasi peran ini memberikan tantangan tersendiri sekaligus memperkaya pengalaman pelaksana dalam pengembangan perangkat lunak end-to-end.