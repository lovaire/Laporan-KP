%----------------------------------------------------------------------------%
\chapter{PENDAHULUAN}
\label{bab:pendahuluan}
%----------------------------------------------------------------------------%

Bab ini menjelaskan latar belakang pelaksanaan kerja praktik, proses pencarian tempat kerja praktik, serta gambaran umum mengenai tempat dan posisi pelaksanaan kerja praktik.

%----------------------------------------------------------------------------%
\section{Proses Pencarian Kerja Praktik}
\label{sec:prosesKP}
%----------------------------------------------------------------------------%

Proses pencarian kerja praktik bermula dari kebutuhan Pondok Modern Al-Ghozali akan sistem manajemen internal yang lebih terstruktur, khususnya pada pengelolaan absensi dan perizinan guru. Berdasarkan kebutuhan tersebut, pelaksana kerja praktik mengajukan proposal pengembangan sistem informasi yang berfokus pada digitalisasi proses absensi dan perizinan.

Proposal kerja praktik disampaikan kepada pihak yayasan dan penyelia dengan menjelaskan ruang lingkup pekerjaan, teknologi yang akan digunakan, serta estimasi waktu pengerjaan. Setelah melalui tahap diskusi dan penyesuaian lingkup pekerjaan, proposal tersebut memperoleh persetujuan dari pihak Pondok Modern Al-Ghozali. Pelaksana kemudian resmi memulai kegiatan kerja praktik sesuai dengan jadwal yang telah disepakati.

%----------------------------------------------------------------------------%
\section{Tempat Kerja Praktik}
\label{sec:tempatKP}
%----------------------------------------------------------------------------%

%----------------------------------------------------------------------------%
\subsection{Profil Tempat Kerja Praktik}
\label{sec:profilTempatKP}
%----------------------------------------------------------------------------%

Pondok Modern Al-Ghozali merupakan lembaga pendidikan formal yang berlokasi di Gunungsindur, Kabupaten Bogor. Lembaga ini menyelenggarakan pendidikan pada jenjang Madrasah Ibtidaiyah (MI), Sekolah Menengah Pertama (SMP), dan Sekolah Menengah Atas (SMA) berbasis pendidikan Islam.

Dalam kegiatan operasionalnya, Pondok Modern Al-Ghozali memiliki beberapa unit pendukung, salah satunya adalah Unit Media dan Teknologi Informasi yang bertanggung jawab dalam pengelolaan dokumentasi serta pengembangan sistem informasi internal. Seiring dengan meningkatnya kebutuhan administrasi berbasis data, pengembangan sistem informasi menjadi salah satu prioritas untuk meningkatkan efisiensi dan akurasi pengelolaan data sekolah.

%----------------------------------------------------------------------------%
\subsection{Posisi Penempatan Pelaksana Kerja Praktik}
\label{sec:posisiKP}
%----------------------------------------------------------------------------%

Pada pelaksanaan kerja praktik ini, pelaksana menempati posisi sebagai Project Manager sekaligus Fullstack Developer. Sebagai Project Manager, pelaksana bertanggung jawab dalam perencanaan timeline proyek, pengelolaan sprint, serta koordinasi dengan pihak penyelia dan tim teknis. 

Sebagai Fullstack Developer, pelaksana bertugas mengembangkan sistem backend, mengolah data absensi dari mesin fingerprint, serta mengintegrasikan sistem perizinan guru berbasis web. Peran ganda ini memberikan pengalaman langsung dalam mengelola proyek pengembangan perangkat lunak sekaligus mengimplementasikan solusi teknis sesuai dengan kebutuhan pengguna.
