%-----------------------------------------------------------------------------%
\chapter{ISI}
\label{bab:2}
%-----------------------------------------------------------------------------%

%-----------------------------------------------------------------------------%
\section{Latar Belakang Pekerjaan}
\label{sec:latarBelakangPekerjaan}
%-----------------------------------------------------------------------------%
Sistem absensi di Pondok Modern Al-Ghozali sebelumnya masih dilakukan secara manual.
Pengolahan data kehadiran yang bersumber dari mesin \textit{fingerprint} memerlukan waktu yang cukup lama karena format file yang dihasilkan tidak standar.
Selain itu, guru juga mengalami kesulitan dalam melakukan koordinasi perizinan dengan guru piket karena proses penyampaian informasi masih dilakukan secara konvensional.

Permasalahan tersebut berdampak pada keterlambatan rekapitulasi data kehadiran serta kurang optimalnya alur perizinan guru.
Oleh karena itu, pembangunan sistem berbasis web berupa dashboard absensi otomatis dan sistem perizinan terintegrasi menjadi solusi utama untuk meningkatkan efisiensi dan akurasi pengelolaan data kehadiran.

%-----------------------------------------------------------------------------%
\section{Deskripsi Pekerjaan}
\label{sec:deskripsiPekerjaan}
%-----------------------------------------------------------------------------%
Pelaksana menjalankan tugas manajerial dan teknis secara bersamaan selama kerja praktik.
Tugas manajerial meliputi penyusunan \textit{timeline} proyek, pengelolaan \textit{backlog}, serta koordinasi melalui pelaksanaan \textit{sprint} harian.
Pelaksana juga berperan dalam memimpin \textit{stand-up meeting} untuk memantau progres dan kendala teknis tim.

Dari sisi teknis, pelaksana bertanggung jawab mengembangkan logika pemrosesan file berformat XLS yang berasal dari mesin absensi menggunakan \textit{library} Pandas.
Selain itu, pelaksana membangun modul dashboard yang menampilkan statistik kehadiran secara \textit{real time}.
Pelaksana juga mengembangkan fitur perizinan guru yang terintegrasi dengan notifikasi WhatsApp untuk guru piket.
Seluruh pengembangan aplikasi dilakukan menggunakan bahasa pemrograman Python dengan framework Django.

%-----------------------------------------------------------------------------%
\section{Tinjauan Pustaka}
\label{sec:tinjauanPustaka}
%-----------------------------------------------------------------------------%
Pekerjaan ini menerapkan konsep pengembangan aplikasi web berbasis framework.
Django merupakan framework Python yang mengadopsi pola \textit{Model-View-Template} (MVT) untuk memisahkan logika bisnis, tampilan, dan pengelolaan data.
Menurut dokumentasi resminya, Django dirancang untuk mempercepat proses pengembangan aplikasi web yang aman dan terstruktur.

Selain itu, \textit{library} Pandas digunakan untuk melakukan manipulasi dan analisis data terstruktur, khususnya dalam pemrosesan data absensi.
Metodologi Scrum diterapkan sebagai pendekatan manajemen proyek untuk mengelola pekerjaan dalam iterasi singkat dan fleksibel, sehingga kebutuhan \textit{stakeholder} dapat diakomodasi secara bertahap.

%-----------------------------------------------------------------------------%
\section{Metodologi}
\label{sec:metodologi}
%-----------------------------------------------------------------------------%
Pengembangan sistem dilakukan dengan menggunakan kerangka kerja Scrum.
Proses diawali dengan pelaksanaan \textit{sprint planning} di awal setiap minggu untuk menentukan target pekerjaan.
Selama sprint berlangsung, pelaksana memimpin \textit{daily stand-up meeting} guna memantau progres dan mengidentifikasi hambatan teknis.

Pada akhir setiap sprint, tim melakukan \textit{sprint review} untuk mendemonstrasikan fitur yang telah dikembangkan.
Pendekatan ini memastikan proses pengembangan berjalan terarah serta memungkinkan penyesuaian fitur berdasarkan kebutuhan pengguna.

%-----------------------------------------------------------------------------%
\section{Teknologi}
\label{sec:teknologi}
%-----------------------------------------------------------------------------%
Proyek kerja praktik ini menggunakan teknologi modern sebagai berikut.
Bahasa pemrograman utama yang digunakan adalah Python versi 3.10.
Framework backend yang digunakan adalah Django versi 5.0.
Penyimpanan data menggunakan basis data MySQL.
Pengembangan antarmuka pengguna memanfaatkan Bootstrap dan JavaScript dengan \textit{library} Chart.js.
Aplikasi dideploy ke server hosting Rumahweb menggunakan konfigurasi cPanel.

%-----------------------------------------------------------------------------%
\section{Hasil Pekerjaan}
\label{sec:hasilPekerjaan}
%-----------------------------------------------------------------------------%
Sistem informasi berhasil dikembangkan dan diimplementasikan dengan fitur yang lengkap.
Fitur unggah file XLS mampu memproses data absensi \textit{fingerprint} dalam waktu singkat.
Dashboard utama menampilkan visualisasi tren kehadiran bulanan dalam bentuk grafik.
Modul perizinan memungkinkan guru mengunggah bukti surat dokter secara digital.
Integrasi notifikasi WhatsApp berhasil mempercepat alur penyampaian informasi antara guru yang mengajukan izin dan guru piket.

%-----------------------------------------------------------------------------%
\section{Aspek Non Teknis}
\label{sec:aspekNonTeknis}
%-----------------------------------------------------------------------------%
Selama pelaksanaan kerja praktik, pelaksana memperoleh berbagai pembelajaran \textit{soft skill}.
Kepemimpinan tim menjadi aspek utama yang diasah melalui pengelolaan sprint.
Komunikasi dengan \textit{stakeholder} membantu pelaksana memahami cara menyampaikan kendala teknis kepada pihak non-teknis.
Manajemen waktu juga menjadi keterampilan penting untuk menyeimbangkan pengembangan sistem dan tugas administratif proyek.

%-----------------------------------------------------------------------------%
\section{Analisis Pelaksanaan Kerja Praktik}
\label{sec:analisisKP}
%-----------------------------------------------------------------------------%

\subsection{Ulasan Kesesuaian dengan KAKP}
Pelaksanaan kerja praktik berjalan sesuai dengan rencana yang tercantum dalam dokumen KAKP.
Realisasi pekerjaan mingguan mengikuti target sprint yang telah ditetapkan.
Fitur unggah data dan dashboard statistik berhasil diselesaikan pada pertengahan periode kerja praktik.
Penambahan fitur notifikasi WhatsApp dilakukan sebagai bentuk improvisasi berdasarkan kebutuhan mendesak di lapangan.

\subsection{Ulasan Kendala dan Cara Menanganinya}
Kendala utama yang dihadapi adalah format data mentah dari mesin \textit{fingerprint} yang tidak konsisten.
Permasalahan ini diatasi dengan membuat skrip \textit{preprocessing} data menggunakan \textit{library} Pandas.
Kendala lain berupa keterbatasan memori pada server hosting Rumahweb diselesaikan dengan melakukan optimasi \textit{query} dan pembersihan file statis.
Seluruh kendala teknis dapat diatasi sebelum masa kerja praktik berakhir.

\subsection{Penilaian Individu Terhadap Tempat Kerja Praktik}
Pondok Modern Al-Ghozali menyediakan lingkungan kerja yang kondusif dan mendukung.
Penyelia lapangan memberikan bimbingan teknis serta fleksibilitas dalam pengembangan sistem.
Rekan tim menunjukkan dedikasi tinggi dalam menyelesaikan tugas masing-masing.
Lingkungan kerja ini sangat ideal bagi mahasiswa untuk menerapkan ilmu yang diperoleh selama perkuliahan.

\subsection{Relevansi dengan Perkuliahan}
Pelaksanaan kerja praktik memiliki relevansi yang kuat dengan materi perkuliahan.
Mata kuliah Basis Data membantu dalam perancangan tabel guru dan jadwal.
Mata kuliah Proyek Perangkat Lunak menjadi dasar penerapan metodologi Scrum.
Konsep pemrograman dari mata kuliah Pemrograman Lanjut mendukung pengembangan backend yang terstruktur dan efisien.
Pengalaman kerja praktik ini memperkuat pemahaman teoretis yang diperoleh selama studi di Fasilkom UI.
