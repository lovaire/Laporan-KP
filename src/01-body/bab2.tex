%-----------------------------------------------------------------------------%
\chapter{ISI}
\label{bab:2}
%-----------------------------------------------------------------------------%
Pada bab ini diuraikan secara terperinci mengenai seluruh aktivitas pekerjaan yang dilaksanakan oleh peserta kerja praktik selama pelaksanaan program kerja praktik. Mulai dari latar belakang pelaksanaan pekerjaan hingga hasil yang diperoleh
%-----------------------------------------------------------------------------%
\section{Latar Belakang Pekerjaan}
\label{sec:latarBelakangPekerjaan}
%-----------------------------------------------------------------------------%
Pelaksana beserta kelompok ditempatkan pada bagian Pengembangan Dashboard Pengelolaan Absensi dan Sistem Perizinan Guru Pondok Modern Al-Ghozali pada tim pengembang TI Pondok Modern Al-Ghozali. Berperan sebagai Project Manager, pelaksana bertanggung jawab atas berjalannya kelancaran seluruh kegiatan Kerja Praktik ini dari awal hingga akhir.

Sistem rekapitulasi presensi di Pondok Modern Al-Ghozali sebelumnya masih manual. Meski sudah menggunakan sistem \textit{fingerprint}, namun untuk pengolahan data supaya menjadi suatu laporan rutin itu masih dilakukan secara manual, dimana guru piket menarik data dari mesin \textit{fingerprint} yang kemudian diserahkan kepada Kepala Bidang (Kabid) Pengajaran selaku pemegang tanggung jawab sekaligus pengelola sistem presensi. Pengolahan data dari mesin \textit{fingerprint} memakan waktu lama karena format file yang tidak standar yakni .xls dimana ini merupakan format versi lama.

%-----------------------------------------------------------------------------%
\section{Deskripsi Pekerjaan}
\label{sec:deskripsiPekerjaan}
%-----------------------------------------------------------------------------%
Pelaksana melakukan tugas manajerial dan teknis secara bersamaan. Tugas manajerial meliputi penyusunan \textit{timeline} proyek dan koordinasi \textit{sprint} dengan seluruh anggota kelompok. Tugas teknis meliputi pengembangan logika pemrosesan file XLS menggunakan \textit{library} Pandas dan sebagainya. Pelaksana membangun modul dashboard yang menampilkan statistik presensi guru dari mesin fingerprint ke dalam bentuk grafik garis dan juga \textit{leaderboard} ketepatan waktu guru mengajar. Seluruh kode aplikasi menggunakan bahasa pemrograman Python dengan \textit{framework} Django. Kerja praktik dilakukan dengan skema \textit{Work from Anywhere} (WFA) dengan jam kerja sesuai yang sudah tertera pada dokumen Kerangka Acuan Kerja Praktik.

%-----------------------------------------------------------------------------%
\section{Tinjauan Pustaka}
\label{sec:tinjauanPustaka}
%-----------------------------------------------------------------------------%
Selama pelaksanaan kerja praktik, pengembangan sistem didasarkan pada berbagai konsep dan teori dalam bidang rekayasa perangkat lunak. Secara khusus, landasan teori yang digunakan berfokus pada pengembangan perangkat lunak berbasis web dari sisi \textit{backend}, pengelolaan data, serta manajemen proyek perangkat lunak. Tinjauan pustaka ini bertujuan untuk memberikan dasar teoretis terhadap pekerjaan yang dilakukan selama kerja praktik berlangsung.

\subsection{Backend Development}
\textit{Backend development} merupakan salah satu bidang dalam pengembangan perangkat lunak yang berfokus pada pengolahan data, logika bisnis, serta komunikasi antara sistem dan basis data. Berbeda dengan \textit{frontend} yang berinteraksi langsung dengan pengguna, \textit{backend} bekerja di balik layar untuk memastikan seluruh proses sistem berjalan dengan benar dan konsisten \cite{hubli2023backend}.

Menurut Hubli dan Jaiswal \cite{hubli2023backend}, \textit{backend development} mencakup perancangan arsitektur sistem, pengelolaan basis data, pengembangan API, serta penerapan aturan bisnis yang mendukung kebutuhan aplikasi. \textit{Backend} memiliki peran penting dalam menjaga integritas data, keamanan sistem, dan performa aplikasi secara keseluruhan.

\subsection{Framework Django}
Django merupakan \textit{framework} pengembangan web berbasis bahasa pemrograman Python yang menerapkan pola arsitektur \textit{Model-View-Template} (MVT). \textit{Framework} ini dirancang untuk mempercepat proses pengembangan aplikasi web dengan menyediakan berbagai fitur bawaan seperti sistem autentikasi, pengelolaan basis data, serta mekanisme keamanan \cite{django2025documentation}.

Menurut dokumentasi resmi Django, penggunaan \textit{framework} ini memungkinkan pengembang untuk membangun aplikasi web yang aman, terstruktur, dan mudah dikembangkan lebih lanjut. Django juga mendukung prinsip \textit{rapid development}, sehingga sangat sesuai digunakan dalam proyek dengan waktu pengembangan yang terbatas seperti kerja praktik. Dalam proyek ini, Django digunakan sebagai fondasi \textit{backend} untuk mengelola logika bisnis sistem absensi dan perizinan guru.

\subsection{Project Management}
Menurut \textit{Project Management Institute} \cite{pmi2021pmbok}, \textit{project management} mencakup proses inisiasi, perencanaan, eksekusi, monitoring, dan penutupan proyek.

Menurut \textit{Project Management Institute} (PMI), \textit{project management} mencakup proses inisiasi, perencanaan, eksekusi, monitoring, dan penutupan proyek. Dalam kerja praktik ini, konsep \textit{project management} diterapkan untuk mengatur alur pengembangan sistem, mengoordinasikan tim, serta memastikan setiap kebutuhan \textit{stakeholder} dapat terpenuhi sesuai dengan jadwal yang telah direncanakan.

%-----------------------------------------------------------------------------%
\section{Metodologi Pekerjaan}
\label{sec:metodologi}
%------------------------------------------------------------------------------

Pelaksanaan kerja praktik ini menerapkan metodologi pengembangan perangkat lunak Scrum, yang merupakan bagian dari pendekatan Agile. Metodologi Scrum dipilih karena mampu mendukung pengembangan sistem secara iteratif serta adaptif terhadap perubahan kebutuhan dari pihak \textit{stakeholder}. Seluruh proses pengembangan sistem dilakukan dalam beberapa siklus pengembangan yang disebut sebagai \textit{sprint}.  

Pada proyek kerja praktik ini, metodologi Scrum digunakan sebagai kerangka kerja utama untuk mengatur alur pengembangan sistem absensi guru. Setiap tahapan pengembangan dilakukan secara bertahap dan berurutan agar hasil yang diperoleh sesuai dengan kebutuhan pihak sekolah. Adapun tahapan metodologi pekerjaan yang dilakukan selama kerja praktik ini dijelaskan sebagai berikut.

%------------------------------------------------------------------------------
\subsection{Pengumpulan Data}
\label{subsec:pengumpulan-data}
%------------------------------------------------------------------------------

Tahapan pertama yang dilakukan pada kerja praktik ini adalah pengumpulan data. Data yang dikumpulkan menjadi dasar utama dalam proses pengembangan sistem absensi guru. Salah satu data utama yang dikumpulkan adalah data hasil pencatatan mesin \textit{fingerprint} yang digunakan oleh pihak sekolah. Data tersebut digunakan sebagai sumber utama untuk mengetahui pola kehadiran guru secara aktual di lapangan.

Selain data \textit{fingerprint}, pelaksana juga mengumpulkan data identitas seluruh tenaga pendidik dan staf, yang mencakup guru pengajar, guru piket, serta staf administrasi. Data ini diperlukan agar sistem dapat mengenali setiap individu yang melakukan pencatatan kehadiran. Pelaksana juga mengumpulkan data jadwal mengajar seluruh guru, termasuk hari dan jam mengajar masing-masing guru. Data jadwal ini berperan penting sebagai acuan dalam menentukan status kehadiran guru berdasarkan waktu \textit{scan} yang tercatat pada mesin \textit{fingerprint}.  

Proses pengumpulan data dilakukan melalui koordinasi dengan pihak sekolah yakni Kepala Bidang (Kabid). Seluruh data yang diperoleh kemudian dikaji kembali untuk memastikan kelengkapan dan konsistensinya sebelum digunakan pada tahap selanjutnya.

%------------------------------------------------------------------------------
\subsection{Pembuatan Master Data}
\label{subsec:master-data}
%------------------------------------------------------------------------------

Setelah seluruh data yang dibutuhkan berhasil dikumpulkan, tahapan berikutnya adalah pembuatan \textit{master data}. Master data berfungsi sebagai referensi utama dalam proses pengolahan dan analisis data kehadiran guru. Pada tahap ini, pelaksana menyusun sebuah struktur data yang terorganisir dan mudah diolah oleh sistem.

Master data yang dibuat berisi beberapa atribut penting, seperti nama guru, hari mengajar dan jam mulai mengajar. Data ini disusun dalam format terstruktur agar dapat digunakan sebagai pembanding terhadap data hasil \textit{fingerprint}. Dengan adanya master data, sistem memiliki acuan yang jelas mengenai jadwal resmi setiap guru sehingga proses penentuan status kehadiran dapat dilakukan secara otomatis.

Pembuatan master data ini juga bertujuan untuk meminimalkan kesalahan interpretasi data mentah yang berasal dari mesin \textit{fingerprint}. Data yang sebelumnya terpisah-pisah dan tidak terstandarisasi diolah menjadi satu sumber data yang konsisten dan siap digunakan pada proses pemodelan.

%------------------------------------------------------------------------------
\subsection{Pembuatan Model Pengolahan Absensi}
\label{subsec:model-absensi}
%------------------------------------------------------------------------------

Tahapan selanjutnya adalah pembuatan model pengolahan data absensi guru. Model ini bertugas untuk mengolah data mentah dari mesin \textit{fingerprint} hingga menghasilkan informasi kehadiran yang mudah dipahami. Data-data awal dari banyak mesin \textit{fingerprint} masih berbentuk file terpisah dengan format \textit{.xls}. Oleh karena itu, tahap pertama yang dilakukan adalah mengonversi data tersebut ke dalam format \textit{.csv} agar lebih mudah diproses menggunakan \textit{library} Pandas.

Data mentah yang dihasilkan mesin \textit{fingerprint} memiliki struktur yang kurang terorganisir, di mana satu guru hanya direpresentasikan dalam satu baris data meskipun melakukan \textit{scan} lebih dari satu kali dalam satu hari. Kondisi ini menyulitkan proses analisis kehadiran. Untuk mengatasi hal tersebut, model yang dibuat akan memecah data tersebut sehingga setiap aktivitas \textit{scan} direpresentasikan dalam baris terpisah. Juga data dari banyak file .csv itu digabungkan menjadi satu kesatuan data yang utuh.

Setelah data dirapikan, setiap waktu \textit{scan} dibandingkan dengan data jadwal yang terdapat pada master data. Jika waktu \textit{scan} dilakukan kurang dari atau sama dengan 15 menit setelah jam mengajar yang tercantum pada master data, maka guru tersebut dinyatakan hadir tepat waktu. Apabila waktu \textit{scan} melebihi batas toleransi tersebut, maka status kehadiran dicatat sebagai terlambat. Jika tidak ditemukan data \textit{scan} pada hari mengajar tertentu padahal jadwal tercantum pada master data, maka sistem akan mencatat status kehadiran sebagai tanpa keterangan.

Selain digunakan untuk menentukan status kehadiran guru, hasil pengolahan data dari model ini juga dimanfaatkan sebagai dasar penyajian informasi secara visual pada sistem. Data kehadiran yang telah diklasifikasikan kemudian diolah menjadi bentuk agregat, seperti jumlah kehadiran per hari, tingkat keterlambatan, serta frekuensi ketidakhadiran. Informasi tersebut selanjutnya ditampilkan dalam bentuk grafik \textit{line chart} yang menggambarkan tren kehadiran guru dalam rentang waktu tertentu. Penyajian grafik ini bertujuan untuk memudahkan pihak sekolah dalam memantau pola kehadiran guru secara keseluruhan dari waktu ke waktu.

Selain visualisasi dalam bentuk grafik, sistem juga menampilkan \textit{leaderboard} kehadiran guru. Leaderboard ini menyajikan peringkat guru berdasarkan tingkat kehadiran dan ketepatan waktu dalam menjalankan kewajiban mengajar. Dengan adanya leaderboard, pihak sekolah dapat dengan mudah mengidentifikasi guru dengan tingkat kedisiplinan tinggi maupun guru yang memerlukan perhatian lebih terkait kehadiran. Penyajian data dalam bentuk visual dan peringkat ini diharapkan dapat meningkatkan transparansi informasi serta menjadi bahan evaluasi dalam pengelolaan kehadiran guru di Pondok Modern Al-Ghozali.

Model pengolahan ini dirancang agar seluruh proses penilaian kehadiran dapat dilakukan secara otomatis tanpa perlu campur tangan manual, sehingga mengurangi potensi kesalahan dan mempercepat proses rekapitulasi absensi.

%------------------------------------------------------------------------------
\subsection{Integrasi Sistem dan Deployment}
\label{subsec:integrasi-deployment}
%------------------------------------------------------------------------------

Setelah model pengolahan absensi selesai dikembangkan, tahap selanjutnya adalah integrasi sistem dengan antarmuka pengguna (\textit{frontend}). Pada tahap ini, pelaksana berdengan rekan dari role frontend untuk menampilkan statistik kehadiran guru dengan line chart dan juga tabel \textit{leaderboard}. Proses integrasi dilakukan melalui pertukaran data menggunakan format yang telah disepakati bersama tim frontend.

Setelah proses integrasi berjalan dengan baik, pelaksana melanjutkan ke tahap \textit{deployment} sistem. Sistem yang telah terintegrasi kemudian dipasang pada \textit{python app} yang sudah disiapkan sebelumnya pada panel hosting website sekolah agar dapat digunakan secara langsung oleh pihak sekolah. Pada tahap ini, pelaksana memastikan bahwa aplikasi berjalan dengan stabil dan dapat memproses data dengan baik.

%-----------------------------------------------------------------------------%
\section{Teknologi}
\label{sec:teknologi}
%-----------------------------------------------------------------------------%
Dalam pelaksanaan kerja praktik ini, digunakan berbagai teknologi untuk mendukung proses pengembangan sistem dashboard absensi guru. Teknologi yang digunakan mencakup bahasa pemrograman, \textit{framework} pengembangan, teknologi \textit{frontend}, serta infrastruktur \textit{deployment}. Adapun teknologi yang digunakan adalah sebagai berikut:

\begin{itemize}
	\item \textbf{Python} \\
	Python merupakan bahasa pemrograman tingkat tinggi yang digunakan sebagai bahasa utama dalam pengembangan sistem. Bahasa ini dipilih karena memiliki sintaks yang mudah dipahami, ekosistem \textit{library} yang luas, serta sangat sesuai untuk pengolahan data dan pengembangan aplikasi berbasis web. Dalam kerja praktik ini, Python digunakan untuk membangun logika bisnis dan memproses data absensi guru yang bersumber dari mesin \textit{fingerprint}.

	\item \textbf{Django} \\
	Django merupakan \textit{framework backend} berbasis Python yang menerapkan pola \textit{Model-View-Template} (MVT). Framework ini dipilih karena menyediakan struktur pengembangan yang terorganisasi, fitur keamanan bawaan, serta kemudahan dalam pengelolaan basis data. Django digunakan untuk membangun sisi \textit{backend} aplikasi, termasuk pengelolaan data absensi, pengolahan hasil model, serta penyediaan \textit{endpoint} yang digunakan oleh sistem \textit{frontend}.

	\item \textbf{Bootstrap} \\
	Bootstrap merupakan \textit{framework frontend} berbasis CSS yang digunakan untuk membangun tampilan antarmuka aplikasi. Framework ini menyediakan komponen antarmuka yang responsif dan konsisten, sehingga mempermudah proses pembuatan tampilan dashboard tanpa harus membangun desain dari awal.

	\item \textbf{JavaScript dan Chart.js} \\
	JavaScript digunakan untuk menangani interaksi pada sisi \textit{frontend}, khususnya dalam menampilkan data secara dinamis. Library Chart.js digunakan untuk menampilkan visualisasi data berupa grafik statistik kehadiran guru, seperti \textit{line chart}. Visualisasi ini membantu pengguna dalam memahami pola dan tren kehadiran secara lebih intuitif.

	\item \textbf{Rumahweb dan cPanel} \\
	Proses \textit{deployment} aplikasi dilakukan menggunakan layanan hosting Rumahweb dengan pengelolaan melalui cPanel. cPanel digunakan untuk mengatur file aplikasi, basis data, serta konfigurasi server. Dengan menggunakan layanan ini, aplikasi dapat dijalankan secara \textit{live} dan diakses oleh pihak sekolah melalui jaringan internet.
\end{itemize}

Penggunaan teknologi-teknologi tersebut saling terintegrasi untuk membangun sistem informasi yang stabil, mudah dikembangkan, serta sesuai dengan kebutuhan Pondok Modern Al-Ghozali selama kerja praktik berlangsung.

%-----------------------------------------------------------------------------%
\section{Hasil Pekerjaan}
\label{sec:hasilPekerjaan}
%-----------------------------------------------------------------------------%
Selama periode kerja praktik berlangsung, pelaksana menyelesaikan beberapa pekerjaan utama yang berfokus pada pengembangan sistem dashboard absensi dan sistem perizinan guru di Pondok Modern Al-Ghozali. Pekerjaan yang dilakukan bersifat teknis dan manajerial, menyesuaikan dengan kebutuhan instansi serta permasalahan yang ditemukan di lapangan. Pada bagian ini dijelaskan hasil pekerjaan yang telah diselesaikan selama kerja praktik berlangsung.

\subsection{Pengembangan Modul Upload dan Pemrosesan Data Absensi}
Salah satu pekerjaan utama yang dilakukan adalah pengembangan modul upload data absensi guru yang bersumber dari mesin \textit{fingerprint}. Sebelumnya, data absensi diolah secara manual oleh pihak sekolah karena format file yang dihasilkan oleh mesin \textit{fingerprint} tidak seragam dan sulit untuk diproses secara langsung.

Pelaksana mengembangkan sebuah fitur upload file berformat XLS yang dapat menerima data mentah dari mesin \textit{fingerprint}. File yang diunggah kemudian diproses secara otomatis menggunakan \textit{library} Pandas untuk melakukan pembersihan data (\textit{data preprocessing}), seperti penghapusan data duplikat, penyesuaian format tanggal dan waktu, serta pemetaan data ke struktur basis data yang digunakan sistem.

Dengan adanya modul ini, proses pengolahan data absensi yang sebelumnya memakan waktu lama dapat dilakukan hanya dalam hitungan detik. Data absensi yang telah diproses kemudian disimpan ke dalam database dan siap digunakan oleh modul lain dalam sistem.

\subsection{Pengembangan Dashboard Statistik dan Leaderboard Kehadiran Guru}
Setelah data absensi berhasil diproses dan disimpan, pelaksana mengembangkan modul dashboard statistik kehadiran guru. Modul ini bertujuan untuk membantu pihak sekolah, khususnya Kepala Bidang, dalam memantau tingkat kehadiran guru secara visual dan mudah dipahami.

Dashboard menampilkan berbagai informasi statistik seperti jumlah kehadiran guru per hari, tren kehadiran bulanan, serta rekapitulasi kehadiran dalam bentuk grafik garis. Visualisasi data dibangun menggunakan \textit{library} Chart.js pada sisi \textit{frontend}, sementara data disediakan oleh \textit{backend} Django melalui query terstruktur ke database.

Selain statistik, dashboard juga menyajikan fitur \textit{leaderboard} yang menampilkan peringkat guru berdasarkan tingkat kehadiran dan ketepatan waktu dalam menjalankan kewajiban mengajar. Fitur ini dirancang untuk memberikan insentif kepada guru agar lebih disiplin dalam kehadiran, sekaligus memudahkan pihak sekolah dalam mengidentifikasi guru yang memerlukan perhatian lebih terkait kehadiran.

Hasil dari pengembangan modul ini adalah sebuah dashboard interaktif yang mampu menyajikan data kehadiran guru sehingga pihak sekolah dapat mengambil keputusan dengan lebih cepat dan berbasis data.

%-----------------------------------------------------------------------------%
\section{Aspek Non Teknis}
\label{sec:aspekNonTeknis}
%-----------------------------------------------------------------------------% 
Pada bagian ini dijelaskan pembelajaran non-teknis yang diperoleh pelaksana selama menjalani kerja praktik di Pondok Modern Al-Ghozali. Aspek non-teknis yang dibahas berfokus pada kemampuan kepemimpinan serta komunikasi dengan \textit{stakeholder}, yang memiliki peran penting dalam mendukung kelancaran pengembangan sistem meskipun pelaksanaan kerja praktik dilakukan secara \textit{work from anywhere} (WFA).

\subsection{Kepemimpinan dalam Pengelolaan Proyek}
Selama pelaksanaan kerja praktik, pelaksana berperan sebagai Project Manager yang bertanggung jawab dalam mengoordinasikan jalannya proyek pengembangan sistem. Peran ini menuntut kemampuan kepemimpinan dalam menyusun alur kerja, menentukan prioritas pengembangan, serta memastikan setiap tahapan proyek berjalan sesuai dengan rencana yang telah disepakati.

Meskipun pelaksanaan kerja praktik dilakukan secara penuh dari jarak jauh, pelaksana tetap memimpin proses perencanaan proyek, pengaturan jadwal \textit{sprint}, serta pemantauan progres pengembangan. Pengalaman ini melatih pelaksana untuk mengambil keputusan secara mandiri, bertanggung jawab terhadap hasil pekerjaan, serta menjaga konsistensi arah pengembangan sistem agar tetap sesuai dengan kebutuhan pihak sekolah.

\subsection{Komunikasi dengan Stakeholder pada Tahap Awal dan Presentasi}
Komunikasi dengan \textit{stakeholder} menjadi aspek non-teknis yang sangat penting, terutama pada tahap awal kerja praktik dan saat penyampaian hasil akhir proyek. Pelaksana melakukan koordinasi awal dengan pimpinan sekolah dan kepala bidang untuk menyepakati ruang lingkup kerja praktik, tujuan pengembangan sistem, serta ekspektasi terhadap hasil yang akan dicapai.

Selain itu, pelaksana juga terlibat langsung dalam proses presentasi sistem kepada pihak sekolah. Pada tahap ini, pelaksana berperan dalam menjelaskan alur sistem, fungsi fitur, serta manfaat yang dihasilkan dari pengembangan dashboard absensi guru. Karena sebagian besar \textit{stakeholder} tidak memiliki latar belakang teknis, pelaksana belajar menyampaikan konsep teknis menggunakan bahasa yang sederhana dan mudah dipahami.

Pengalaman komunikasi ini membantu pelaksana memahami pentingnya penyampaian informasi yang jelas dan efektif, sehingga kebutuhan pengguna dapat dipahami dengan baik dan solusi yang dikembangkan dapat diterima serta dimanfaatkan secara optimal oleh pihak sekolah.

%-----------------------------------------------------------------------------%
\section{Analisis Pelaksanaan Kerja Praktik}
\label{sec:analisisKP}
%-----------------------------------------------------------------------------%
Pada bagian ini dipaparkan analisis mengenai pelaksanaan kerja praktik secara menyeluruh. Uraian diawali dengan pembahasan kesesuaian antara pelaksanaan kerja praktik dan kerangka acuan kerja praktik. Selanjutnya, dijelaskan pula berbagai kendala yang ditemui selama proses kerja praktik berlangsung beserta solusi yang diterapkan untuk mengatasinya.
\subsection{Ulasan Kesesuaian dengan KAKP}

\begin{table}[H]
\centering
\caption{Kesesuaian Rencana dan Realisasi Kerja Praktik}
\label{tab:kesesuaian_kakp}
\begin{tabular}{|c|p{7cm}|p{4cm}|}
\hline
\textbf{Minggu} & \textbf{Rencana Kerja} & \textbf{Realisasi Kerja} \\ \hline
1  & Menyusun timeline proyek, menetapkan alur kerja tim, serta memimpin diskusi perencanaan. & Sesuai dengan KAKP \\ \hline
2  & Sprint planning dan pengerjaan backlog & Sesuai dengan KAKP \\ \hline
3  & Review testing dan sprint retrospective & Sesuai dengan KAKP \\ \hline
4  & Sprint planning dan pengerjaan backlog & Sesuai dengan KAKP \\ \hline
5  & Review testing dan sprint retrospective & Sesuai dengan KAKP \\ \hline
6  & Sprint planning dan pengerjaan backlog & Sesuai dengan KAKP \\ \hline
7  & Review testing dan sprint retrospective & Sesuai dengan KAKP \\ \hline
8  & Sprint planning dan pengerjaan backlog & Sesuai dengan KAKP \\ \hline
9  & Review testing dan sprint retrospective & Sesuai dengan KAKP \\ \hline
10 & Sprint planning dan pengerjaan backlog & Sesuai dengan KAKP \\ \hline
11 & Review testing dan sprint retrospective & Sesuai dengan KAKP \\ \hline
12 & Kick-Off proyek dan Presentasi & Sesuai dengan KAKP \\ \hline
13 & On-Live Support & Sesuai dengan KAKP \\ \hline
\end{tabular}
\end{table}

Seperti yang dapat dilihat pada Tabel~\ref{tab:kesesuaian_kakp}, keseluruhan pelaksanaan kerja praktik telah berjalan sesuai dengan rencana yang tertera pada Kerangka Acuan Kerja Praktik (KAKP). Setiap aktivitas yang direncanakan pada tiap minggunya berhasil dilaksanakan dengan tepat. Pelaksana mampu mengelola tugas-tugas yang ada sesuai dengan jadwal yang telah ditetapkan, sehingga seluruh proses pengembangan sistem dapat berjalan lancar tanpa hambatan berarti.


\subsection{Ulasan Kendala dan Cara Menanganinya}
Selama pelaksanaan kerja praktik berlangsung, pelaksana menghadapi beberapa kendala yang memengaruhi proses pengembangan sistem. Kendala yang ditemui tidak hanya bersifat teknis, tetapi juga non-teknis yang berkaitan dengan koordinasi dan ketersediaan informasi dari pihak terkait. Kendala-kendala tersebut menjadi bagian dari dinamika pelaksanaan kerja praktik dan memberikan pembelajaran penting bagi pelaksana dalam mengelola proyek berbasis teknologi informasi di lingkungan institusi pendidikan.

Berikut ini dijelaskan beberapa kendala utama yang dihadapi selama kerja praktik beserta upaya yang dilakukan untuk menanganinya.

\subsubsection{Tidak Adanya File Master Data Jadwal Mengajar Guru}
Salah satu kendala teknis utama yang dihadapi selama kerja praktik adalah tidak tersedianya file master data jadwal mengajar guru dalam format terstruktur. Jadwal mengajar para guru dibuat oleh Kepala Bidang (Kabid) dengan bantuan alat \textit{aSc Timetables}, dimana setelah itu jadwal dikirimkan kepada guru-guru melalui selembar kertas dan juga file PDF untuk masing-masing guru. Sehingga pelaksana harus mencari alternatif lain dalam pengembangan.

Karena keterbatasan sumber daya yang tersedia, untuk mengatasi kendala tersebut, pelaksana terlebih dahulu mengumpulkan data jadwal mengajar dari seluruh guru, kemudian melakukan proses penyusunan ulang data menjadi format yang lebih terstruktur dalam hal ini dalam format CSV. Dalam tahap ini pelaksana merubah satu-persatu jadwal guru yang tadinya berformat PDF, menjadi sebuah file CSV berisikan jadwal jam mengajar seluruh guru yang ada.

\subsubsection{Kepala Bidang yang Sulit Dihubungi untuk Koordinasi}
Kendala non-teknis yang dihadapi selama kerja praktik adalah keterbatasan waktu dan sulitnya melakukan koordinasi langsung dengan Kepala Bidang yang bertanggung jawab terhadap sistem absensi dan perizinan. Sebagai pejabat struktural, Kepala Bidang memiliki banyak tanggung jawab administratif dan operasional, sehingga tidak selalu dapat dihubungi secara langsung dalam waktu singkat.

Kondisi ini berdampak pada proses pengambilan keputusan, khususnya ketika pelaksana membutuhkan konfirmasi terkait alur bisnis, kebijakan perizinan, maupun validasi terhadap fitur yang sedang dikembangkan. Keterlambatan dalam memperoleh respons berpotensi menghambat progres pengembangan sistem apabila tidak ditangani dengan baik.

Untuk mengatasi hal ini, pelaksana membuat asumsi pribadi terlebih dahulu yang nantinya akan diajukan ke Kepala Bidang (Kabid). Selain asumsi diri pelaksana pribadi, pelaksana juga kerap bertanya kepada para guru yang mengajar di Pondok Modern Al-Ghozali, meninjau merekalah yang nantinya akan menjalankan sistem ini kedepannya.

\subsection{Penilaian Individu Terhadap Tempat Kerja Praktik}
Pondok Modern Al-Ghozali memberikan kebebasan dan kepercayaan yang sangat besar kepada pelaksana selama pelaksanaan kerja praktik berlangsung. Sejak awal, pihak sekolah membebaskan pelaksana untuk menentukan pola kerja, metode pengembangan, serta alur penyelesaian proyek sesuai dengan kebutuhan dan kondisi yang ada. Pelaksanaan kerja praktik dilakukan secara penuh dengan skema \textit{work from anywhere} (WFA), tanpa kewajiban kehadiran fisik di lokasi, kecuali pada kegiatan tertentu seperti presentasi dan penyampaian hasil akhir.

Kebebasan yang diberikan tersebut didasarkan pada tingkat kepercayaan pihak sekolah terhadap pelaksana, baik dari sisi latar belakang maupun pemahaman terhadap kebutuhan institusi. Pelaksana yang merupakan alumni Pondok Modern Al-Ghozali telah memahami konteks permasalahan, budaya kerja, serta kondisi lapangan, sehingga proses pengembangan sistem dapat dilakukan secara lebih mandiri dan tepat sasaran. Dalam hal ini, lingkungan kerja praktik terasa lebih sebagai ruang kolaborasi yang terbuka dibandingkan dengan lingkungan kerja yang bersifat hierarkis dan ketat.

Peran penyelia lapangan juga berjalan secara fleksibel. Penyelia tidak menerapkan pengawasan yang kaku, melainkan memberikan kepercayaan penuh kepada pelaksana dalam mengambil keputusan teknis maupun manajerial. Hubungan yang sudah terjalin sebelumnya membuat proses komunikasi berlangsung secara natural dan tidak formal, sehingga diskusi terkait kebutuhan sistem maupun kendala yang dihadapi dapat dilakukan dengan lebih nyaman dan efektif.

Pendekatan kerja yang bebas namun berbasis kepercayaan ini justru mendorong pelaksana untuk bertanggung jawab penuh terhadap hasil kerja yang dihasilkan. Pelaksana dituntut untuk mengatur waktu, menjaga kualitas pengembangan, serta memastikan bahwa sistem yang dibangun benar-benar memberikan manfaat bagi pihak sekolah. Secara keseluruhan, Pondok Modern Al-Ghozali menjadi tempat kerja praktik yang sangat mendukung pengembangan kemandirian, inisiatif, dan tanggung jawab profesional mahasiswa dalam menghadapi permasalahan nyata di dunia kerja.

\subsection{Relevansi dengan Perkuliahan}
Pada pelaksanaan kerja praktik ini, terdapat beberapa tugas yang relevan dengan pembelajaran yang telah diperoleh pelaksana selama menempuh perkuliahan di Fakultas Ilmu Komputer Universitas Indonesia. Mata kuliah yang relevan tersebut menjadi dasar dalam memahami konsep serta menyelesaikan pekerjaan yang dilakukan selama kerja praktik. Adapun mata kuliah yang relevan adalah sebagai berikut:

\begin{itemize}
    \item \textbf{Dasar-Dasar Pemrograman 1} \\
    Bahasa pemrograman utama yang digunakan selama kerja praktik adalah Python. Konsep dasar pemrograman seperti penggunaan variabel, struktur kontrol, perulangan, serta fungsi yang dipelajari pada mata kuliah Dasar-Dasar Pemrograman 1 menjadi fondasi utama dalam memahami dan mengembangkan logika pemrosesan data absensi guru.

    \item \textbf{Pemrograman Berbasis Platform} \\
    Proyek yang dikerjakan selama kerja praktik memanfaatkan \textit{framework} Django sebagai kerangka kerja pengembangan aplikasi \textit{backend}. Konsep pengembangan aplikasi berbasis \textit{framework}, pengelolaan \textit{routing}, pemisahan logika aplikasi, serta penggunaan pola \textit{Model-View-Template} (MVT) yang dipelajari pada mata kuliah Pemrograman Berbasis Platform diterapkan secara langsung dalam proyek ini.

    \item \textbf{Basis Data} \\
    Dalam pengembangan sistem absensi, pelaksana menerapkan konsep perancangan basis data, relasi antar tabel, serta penggunaan \textit{query} untuk mengambil dan mengolah data kehadiran guru. Seluruh proses tersebut berkaitan erat dengan materi yang dipelajari pada mata kuliah Basis Data, khususnya terkait pengelolaan data terstruktur dan integritas data.

    \item \textbf{Kecerdasan Artifisial dan Sains Data Dasar} \\
    Proyek kerja praktik ini melibatkan pembuatan model pengolahan data absensi berbasis data \textit{fingerprint} guru. Proses pengolahan, pembersihan data, serta analisis pola kehadiran menggunakan bahasa pemrograman Python selaras dengan materi yang dipelajari pada mata kuliah Kecerdasan Artifisial dan Sains Data Dasar.
\end{itemize}
